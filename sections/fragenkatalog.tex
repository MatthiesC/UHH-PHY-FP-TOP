\chapter{Fragenkatalog zur Vorbereitung}

\begin{itemize}
\item Warum wird generell eine sehr große Anzahl an Kollisionsereignissen benötigt, um aussagekräftige Teilchenphysik-Studien durchzuführen? -- Würde es beispielsweise für die Bestimmung der Top-Quark-Masse ausreichen, auf ein einzelnes Ereignis zu warten, das ein Top-Quark enthielte, wonach man den LHC abschalten und Feierabend machen könnte? Wenn nicht: Warum wird eine große Anzahl an Ereignissen benötigt, in denen Top-Quarks produziert werden?
\item Warum wird am LHC vornehmlich auf transversale Größen rekonstruierter Objekte zurückgegriffen (\zB{} auf den transversalen Impuls, \pT{})? 
\end{itemize}
