\chapter{Theoretische Grundlagen}

%Dieses Kapitel stellt den wesentlichen theoretischen Hintergrund dar, welcher zum Verständnis dieses Praktikumsversuchs vonnöten ist. In Abschnitt \ref{sec:sm} wird das Standardmodell der Teilchenphysik, welches Ihnen aus der Vorlesung "`Physik 5"' geläufig sein sollte, zusammengefasst. Da sich dieser Versuch im Speziellen mit dem Top-Quark auseinandersetzt, bietet Ihnen Abschnitt \ref{sec:top} einen Überblick über dessen Eigenschaften und über die Top-Quark-Produktionsmechanismen am LHC.

\section{Das Standardmodell der Teilchenphysik}
\label{sec:sm}

Das Standardmodell (SM) der Elementarteilchenphysik bildet den theoretischen Rahmen zur Beschreibung der nach heutigem Kenntnisstand fundamentalen Bausteine der Materie und der Wechselwirkungen zwischen diesen. Das SM ist eine Quantenfeldtheorie, deren Anfänge bis in die 1940er Jahre zurückdatieren und welche in den darauffolgenden Jahrzehnten aus verschiedenen Motivationen heraus mehrfach erweitert worden ist\footnote{Im Speziellen bezieht sich diese Aussage auf die Begründung der Quantenelektrodynamik (QED) durch Richard Feynman \textit{et al.} und die spätere Anwendung der gleichen theoretischen Konzepte auf die starke und schwache Kernkraft in den 1970er Jahren.}. Einerseits waren die Weiterentwicklungen der Theorie begründet durch neue Beobachtungen, die von der experimentellen Teilchenphysik gemacht worden sind, andererseits sind auch viele Fälle zu verzeichnen, in denen die Theorie dem Experiment zuvorgekommen ist. Das heutzutage prominenteste Beispiel für die Vorhersagekraft des SM ist der Nachweis der Existenz des Higgs-Bosons am LHC (vgl. Physik-Nobelpreis 2013).

Trotz des großen Erfolgs des SM ist es nicht in der Lage, sämtliche physikalischen Phänomene zu erklären, die in der Natur beobachtet werden können. So wird beispielsweise Gravitation bisher nicht durch eine Quantenfeldtheorie beschrieben, die mit dem SM vereinigt werden könnte. Des Weiteren beinhaltet das SM weder Teilchen, die als Kandidaten für dunkle Materie infrage kämen, noch liefert es irgendeine Erklärung für die mutmaßliche Existenz dunkler Energie. Diese und weitere Umstände veranlassen Teilchenphysiker dazu, die momentan etablierte Theorie auf die Probe zu stellen (Präzisionstests des SM) und potentielle Erweiterungen des SM experimentell zu überprüfen (Suche nach neuen Teilchen, Wechselwirkungen und anderen Effekten\footnote{Zusammenfassende Oberbegriffe: "`Neue Physik"' bzw. "`Physics \textbf{b}eyond the \textbf{S}tandard \textbf{M}odel"' (BSM).}).

\section{Physik von Proton-Proton-Kollisionen}
\label{sec:pp}

\section{Physik des Top-Quarks}
\label{sec:top}
